\documentclass{mce-article}

\usepackage[T2A]{fontenc}
\usepackage[utf8]{inputenc}
\usepackage[russian]{babel}

\usepackage{amsmath}
\usepackage{graphicx}
\usepackage{url}

\graphicspath{{figures/}}

\begin{document}

\title{Метод моделирования динамики показателей курсовой стоимости
  акций на~российском фондовом рынке}

\author{Парамонов\,А.\,В., Щетинин\,Е.\,Ю.}

\maketitle

\begin{abstract}
  В работе проведена количественная характеризация некоторых
  эмпирических эффектов, сопутствующих процессу изменения курсовой
  стоимости акций на российском фондовом рынке. Предложена
  математическая модель процесса, описываемая стохастическим
  уравнением Фоккера"--~Планка с неоднородным коэффициентом
  диффузии. Показано, каким образом предложенная модель может быть
  использована для решения актуальной практической задачи
  краткосрочного прогноза значений курсовой стоимости акций и биржевых
  показателей.
\end{abstract}

\paragraph{Эмпирические эффекты.}

Процессы изменения финансовых показателей на современных финансовых
рынках обладают сложной, существенно нелинейной структурой. Известно,
что эмпирические распределения приращений финансовых показателей
являются в общем случае асимметричными и обладают степенными хвостами,
а динамика этих процессов характеризуется кластеризацией волатильности
и медленным угасанием автокорелляционной функции квадратов приращений
\cite{Mantegna-Stanley, Muzy}.

Эти эмпирические эффекты свойственны и процессам изменения курсовой
стоимости акций крупных российских компаний. Характерное поведение
эмпирических распределений в области экстремальных значений на
коротких ($\Delta~t~=~\mbox{15 мин}$) временных интервалах подробно
описано в работе \cite{Schetinin:EmpiricalEffects}. Показано, что эти
распределения обладают выраженными степенными хвостами.

Известно также, что на более длинных временных интервалах
($\Delta~t~>~\mbox{1 неделя}$) распределения приближаются к
нормальному. Свойства эмпирических распределений существенно зависят
от выбранного масштаба времени.

Для многих типов финансовых показателей характер зависимости начальных
моментов эмпирических распределений от масштаба времени хорошо
описывается степенным законом \cite{Mantegna-Stanley, Muzy, Peinke}:

\begin{equation}
  \label{eq:multiscaling}
  M_n(\Delta X(\Delta t)) = C_n \cdot (\Delta t)^{\kappa_n}.
\end{equation}

Были расчитаны оценки начальных моментов порядков $n = 1, 2, 4$ для
выборок логарифмических приращений (лог.~доходностей) курсовой
стоимости акций крупных российских компаний, а также индекса ММВБ в
период с~1999 по~2007~гг. Значения временных интервалов взяты в
пределах от 5~минут до 1~месяца. Обнаружено, что во всех случаях
характер зависимости начальных моментов от величины интервала
приращения также хорошо описывается соотношением
\eqref{eq:multiscaling}.

\begin{figure}
  \centering
  \includegraphics{multiscaling-moments}
  \caption{Зависимость моментов эмпирических распределений
    лог.~приращений от масштаба времени. Линиями обозначена
    аппроксимация степенной функцией.}
\end{figure}

% \begin{table}
%   \centering
%   \small
%   \begin{tabular}{l|ccc}
%     &$\kappa_1$&$\kappa_2$&$\kappa_4$\\\hline
%     EESR &     0.99 &     1.07 &     2.02 \\
%     GMKN &     1.00 &     1.04 &     1.71 \\
%     LKOH &     1.00 &     1.02 &     1.77 \\
%     RTKM &     0.99 &     1.08 &     1.92 \\
%     MICEX&     1.00 &     1.05 &     1.59 \\
%   \end{tabular}
%   \caption{Численные оценки значений параметров $\kappa_n$ для
%     распределений лог.~приращений стоимости акций крупных
%     российских компаний и фондового индекса ММВБ.}
%   \label{fig:condprobs}
% \end{table}

Обнаруженная закономерность отражает важные свойства нелинейного
стохастического процесса, порождающего наблюдаемую каскадную структуру
эмпирических распределений. Использование классических
эконометрических методов стохастического анализа, в частности
основанных на авторегрессионых моделях типа GARCH \cite{Bollerslev},
для описания такого рода процессов и структур является
неэффективным. Указанные подходы позволяют с приемлемой точностью
описать распределения для отдельных временных интервалов, но не
связывающие их закономерности \cite{Brummelhuis}.

\paragraph{Математические модели и методы.}

Мы рассматриваем изменение финансовых показателей как \emph{марковский
  процесс}, описываемый стохастическим дифференциальным уравнением
Фоккера"--~Планка \cite{Risken:Fokker-Planck}:

\begin{equation}
  \label{eq:Fokker-Planck}
  \frac{\partial}{\partial \tau} p(x, \tau)
  = \left( - \frac{\partial}{\partial x} D^{(1)}(x, \tau)
    + \frac{\partial^2}{\partial x^2} D^{(2)}(x, \tau)
  \right) p(x, \tau),
\end{equation}

где $x = \Delta X$ "--- значение приращения показателя курсовой
стоимости за интервал времени $\Delta t = \Delta t_0 e^{-\tau}$,

\begin{eqnarray}
  D^{(1)}(x, \tau) &=& -x,\\
  D^{(2)}(x, \tau) &=& \alpha(\tau) + \beta(\tau) \cdot x^2.
\end{eqnarray}

Необходимо обратить внимание на направление времени стохастического
процесса $\tau$, которое в данном случае \emph{противоположно}
естественному направлению времени $\Delta t$.  Уравнение
\eqref{eq:Fokker-Planck} справедливо и для условных функций плотности
распределения $p(x_2, \tau_2| x_1, \tau_1),\, \tau_1 < \tau_2$.

Количественное оценивание параметров модели $\alpha(\tau)$,
$\beta(\tau)$ может быть произведено непосредственно по исходным
данным с использованием значений начальных моментов безусловных
эмпирических распределений. Умножив обе части уравнения
\eqref{eq:Fokker-Planck} на $x^n$ и интегрируя по $x$, получаем:

\begin{eqnarray}
  \label{eq:Kramer-Moyal}
  \frac{\partial}{\partial \tau} M_n(x(\tau))
  &=& \sum\limits_{k=1}^{2} \left( -1 \right)^k 
  \int\limits_{-\infty}^{+\infty}  x^n  \left( 
    \frac{\partial}{\partial x}  \right)^k  D^{(k)}(x,\tau)  
  p(x,\tau) dx \nonumber\\
  &=& \sum\limits_{k=1}^{2}  \frac{n!}{(n - k)!} 
  \int\limits_{-\infty}^{+\infty}  x^{n-k}  D^{(k)}(x,\tau)  
  p(x,\tau) dx.
\end{eqnarray}

Подставляя выражения для $D^{(1)}(x, \tau)$, $D^{(2)}(x, \tau)$ и
используя сокращенное обозначение $M_n(\tau) = M_n(x(\tau))$, имеем

\begin{eqnarray}
  \frac{\partial}{\partial \tau} M_2(\tau)
  &=& -2 M_2(\tau) + 2 \alpha(\tau) + 2 \beta(\tau) M_2(\tau),\\
  \frac{\partial}{\partial \tau} M_4(\tau)
  &=& -4 M_4(\tau) + 12 \alpha(\tau) M_2(\tau)
  + 12 \beta(\tau) M_4(\tau).
\end{eqnarray}

Используя выражения для $M_n$, записанные в координатах
стохастического процесса, получаем:

\begin{eqnarray}
  \label{eq:alphabetaM2slae}
  - \kappa_2 M_2(\tau) &=& -2 M_2(\tau) +  2 \alpha(\tau)
  +  2 \beta(\tau) M_2(\tau),\\
  \label{eq:alphabetaM4slae}
  - \kappa_4 M_4(\tau) &=& -4 M_4(\tau) + 12 \alpha(\tau) M_2(\tau)
  + 12 \beta(\tau) M_4(\tau).
\end{eqnarray}

Система линейных алгебраических уравнений
\eqref{eq:alphabetaM2slae}--\eqref{eq:alphabetaM4slae} позволяет найти
значения функций $\alpha$, $\beta$ для каждого момента времени $\tau$.
Ниже приведено решение системы.

\begin{eqnarray}
  \alpha(\tau) &=& \frac{M_2(\tau)}{12}
  \cdot \frac{(8 - 6 \kappa_2 + \kappa_4) M_4(\tau)}
  {M_4(\tau) - M_2^2(\tau)},\\
  \beta(\tau) &=& \frac{1}{12}
  \cdot \frac{(4 - \kappa_4) M_4(\tau)
    + (-12 + 6 \kappa_2) M_2^2(\tau)}
  {M_4(\tau) - M_2^2(\tau)}.
\end{eqnarray}

Решение уравнения \eqref{eq:Fokker-Planck} было произведено численно
методом Монте"--~Карло. По функциям распределения 2-суточных
логарифмических приращений реконструированы распределения для более
коротких временных интервалов (вплоть до 15~минут). Во всех случаях
обнаружено хорошее соответствие между эмпирическим и предсказанным
распределениями.

\begin{figure}
  \centering
  \includegraphics{MICEX-simulation}
  \caption{Реконструкция распределений лог.~приращений фондового
    индекса ММВБ. Для облегчения восприятия профили функций плотности
    распределения смещены по оси ординат.}
\end{figure}

Параметры модели $D^{(1)}(x, \tau)$, $D^{(2)}(x, \tau)$ являются
симметричными функциями относительно $x$. Тем не менее, предложенная
модель позволяет получать асимметричные функции распределения, в
случае если начальное условие является асимметричным.

Таким образом, предложенная модель позволяет естественным образом
описать наблюдаемые эмпирические эффекты:

\begin{enumerate}
\item Асимметрия распределений приращений финансовых показателей;
\item Тяжелые (степенные) хвосты распределений на малых временных
  интервалах;
\item Лёгкие (экспоненциальные) хвосты распределений на больших
  временных интервалах.
\end{enumerate}

\begin{figure}
  \centering
  \begin{minipage}[t]{0.45\textwidth}
    \centering
    \includegraphics{MICEX-qqplot}
    \caption{Квантили эмпирического и реконструированного
      распределений 15-минутных лог.~приращений фондового индекса
      ММВБ.}
  \end{minipage}
  \hspace{0.05\textwidth}
  \begin{minipage}[t]{0.45\textwidth}
    \centering
    \includegraphics{MICEX-predict-logprobs}
    \caption{Условное распределение 5-минутных лог.~приращений
      фондового индекса ММВБ.}
    \label{fig:condprobs}
  \end{minipage}
\end{figure}

\paragraph{Краткосрочный прогноз.}

Математический прогноз заключается в построении условной функции
распределения будущих значений финансового показателя по известным
значениям в прошлом. Напомним, что рассматриваемый стохастический
процесс по предположению является марковским, т.е.

\begin{equation}
  p(x_n, \tau_n| x_{n - 1}, \tau_{n - 1}, \ldots, x_1, \tau_1)
  = p(x_n, \tau_n| x_{n - 1}, \tau_{n - 1}),
  \label{eq:Markov-property}
\end{equation}

для всех $x_1, x_2, \ldots, x_n$, $\tau_1 < \ldots < \tau_{n - 1} <
\tau_n$. Формула полной вероятности для марковского процесса имеет вид

\begin{equation}
  p(x_n, \tau_n, \ldots, x_1, \tau_1)
  = p(x_n, \tau_n| x_{n - 1}, \tau_{n - 1})
  \cdot p(x_{n - 1}, \tau_{n - 1}, \ldots, x_1, \tau_1).
  \label{eq:total-prob}
\end{equation}

Рекурсивно используя \eqref{eq:total-prob}, получаем формулу для
расчёта вероятности изменения значения финансового показателя:

\begin{equation}
  p(x_n, \tau_n, \ldots, x_1, \tau_1)
  = p(x_n, \tau_n| x_{n - 1}, \tau_{n - 1}) \cdot \ldots
  \cdot p(x_2, \tau_2| x_1, \tau_1) \cdot p(x_1, \tau_1).
\end{equation}

На Рис.~\ref{fig:condprobs} приведен профиль условной функции
плотности распределения 5-минутных логарифмических приращений
фондового индекса ММВБ, расчитанный по 12~последовательным значениям
данного финансового показателя за предшествующий час (выборка от
24~декабря 2007~г.). В настоящей работе значения условных вероятностей
$p(x_k, \tau_k| x_{k - 1}, \tau_{k - 1})$ получены путём численного
решения уравнения \eqref{eq:Fokker-Planck}.

\begin{thebibliography}{99}
\bibitem{Mantegna-Stanley} \textit{R.\,Mantegna, H.\,E.\,Stanley} An
  introduction to Econophysics\ // Cambridge University Press. 1999.
\bibitem{Muzy} \textit{A.\,Arneodo, J."--~F.\,Muzy, D.\,Sornette}
  Causal cascade in the stock market from the ``infrared'' to the
  ``ultraviolet''\ // European Physical Journal
  B. 1998. Vol.\,2. P.\,277--282.
\bibitem{Schetinin:EmpiricalEffects} \textit{Щетинин\,Е.\,Ю.}
  Статистический анализ свойств структур экстремальных зависимостей на
  российском фондовом рынке\ // Финансы и кредит.
  2005. №22(190). С.\,44--51.
\bibitem{Peinke} \textit{Ch.\,Renner, J.\,Peinke, R.\,Friedrich}
  Evidence of Markov properties of high frequency exchange rate data.\
  // Physica A. 2001.\,3. Vol.\,298. P.\,499--520(22).
\bibitem{Bollerslev} \textit{T.\,Bollerslev} Generalized
  autoregressive conditional heteroskedasticity\ // Journal of
  Econometrics. 1986. Vol.\,31. P.\,307--327.
\bibitem{Brummelhuis} \textit{R.\,Brummelhuis, D.\,Guegan} The Time
  Scaling of Value-at-Risk in GARCH(1,1) and AR(1)"--~GARCH(1,1)
  Processes\ // Journal of Risk. 2007. Vol.\,9(4). P.\,39--94.
\bibitem{Risken:Fokker-Planck} \textit{H.\,Risken} The Fokker-Planck
  equation\ // Springer-Verlag. 1984.
\end{thebibliography}

\bigskip

\title{Modelling the dynamics of price fluctuations at~Russian stock
  market}

\author{Paramonov\,A.\,V., Shchetinin\,Ye.\,Yu.}

\maketitle

\begin{abstract}
  The price fluctuations at Russian stock market are described by a
  number of empirical effects or ``stylized facts''. We perform a
  quantitative characterization of some of these effects, and propose
  a model of the underlying stochastic process based on nonlinear
  Fokker-Planck equation. We show how the model can be applied for
  short-term forecast of stock prices and market index values.
\end{abstract}

\end{document}
